\documentclass[aspectratio=169,mathserif,xcolor=dvipsnames]{beamer}
 
\useoutertheme[numbering=fraction]{metropolis}
\useinnertheme{metropolis}
\usefonttheme{metropolis}
\usecolortheme{default}

\usepackage{array}
\usepackage{subcaption}
\usepackage{graphicx}
\usepackage[longnamesfirst]{natbib}
\usepackage{minibox}
\usepackage{tabularx}

\input{../../header-files/src/metropolis-titlepage}
\input{../../header-files/src/header.tex}
\input{../../header-files/src/acronyms.tex}
\input{../../header-files/src/math-def.tex}
\input{../../header-files/src/tikz-def.tex}
\input{../../header-files/src/beamer-header.tex}
\input{../../header-files/src/falling-behind.tex}

\usepackage{pgfplots}
\usetikzlibrary{tikzmark, positioning}

\definecolor{LRed}{rgb}{1,.8,.8}
% alert color
% \renewcommand{\alert}[1]{{\usebeamercolor[fg]{frametitle}{#1}}}

% customize color commands
% \newcommand{\red}[1]{\textcolor{red}{#1}}
\newcommand{\maroon}[1]{\textcolor{Maroon}{#1}}
% \newcommand{\blue}[1]{\textcolor{blue}{#1}}
% \newcommand{\green}[1]{\textcolor{green}{#1}}
%\newcommand{\gray}[1]{\textcolor{gray}{#1}}

%\newcommand{\eps}{\varepsilon}

\AtBeginSection[]
{
  \frame{\sectionpage}
  % \begin{frame}
  %   \frametitle{Outline}
  %   \tableofcontents[currentsubsection]
  % \end{frame}
}

%\usepackage[longnamesfirst]{natbib}

% \usepackage[ngerman]{babel}
%\usepackage[applemac]{inputenc}

%\renewcommand{\caption}[1]{}

%\title{The Distribution of Household Debt and Financial Fragility}
\title{Essays on the Macroeconomics of Housing Markets}


\author{Fabian Greimel}
\institute[U Mannheim]{\vspace{-10pt} University of Mannheim}
\date{\vspace{10pt} Dissertation Defense $|$ Mannheim  $|$ August 19, 2020}


\begin{document}
 
 \frame[plain]{\maketitle}


 \begin{frame}{Outline}

   \begin{description}
   \item[Chapter 1] Falling Behind: Has Rising Inequality \\ Fueled the American Debt Boom?
   \item[Chapter 2] Top Incomes And Mortgage Debt Across the United States
   \item[Chapter 3] Understanding Housing Wealth Effects: \\ Debt, Home-Ownership and the Lifecycle
   \end{description}
 \end{frame}
 
 \section{Chapter 1 \\ Falling Behind: Has Rising Inequality \\ Fueled the American Debt Boom \\ \footnotesize{with Moritz Drechsel-Grau}}

 \subsection{Introduction}
 \input{../../falling-behind/slides/facts}
 This dissertation consists of three chapters. Each chapter is self-contained.

\paragraph{Chapter 1} is joint work with Moritz Drechsel-Grau.\footnote{University of Mannheim.}
%\input{../../falling-behind/paper/abstract}

\paragraph{Chapter 2} is joint work with Moritz Drechsel-Grau as well.
%\input{../../inequality-mortgages/paper/abstract}

\paragraph{Chapter 3} is joint work with Frederick Zadow.\footnote{University of Mannheim.}
%\input{../../housing-wealth-effects/paper-old/abstract}



%%% Local Variables:
%%% TeX-engine: luatex
%%% TeX-master: "main"
%%% End:

 \input{../../falling-behind/slides/mechanism}
 \subsection{Model}
 \input{../../falling-behind/slides/model}
 \subsection{Quantitative Results}
 \input{../../falling-behind/slides/inequality-experiment}
 %\input{../../falling-behind/slides/horse-race}
 \subsection{Conclusion}
 \input{../../falling-behind/slides/outlook}

 
 \section{Chapter 2 \\ Top Incomes And Mortgage Debt Across the United States \\ \footnotesize{with Moritz Drechsel-Grau}}

 \begin{frame}{In A Nutshell}

   \begin{block}{Questions}
     \begin{enumerate}
     \item Is there an empirical relationship between debt and and inequality?
     \item Can we learn something about the underlying mechanism?
     \end{enumerate}
   \end{block}
\pause
   \begin{block}{Data}
     \begin{itemize}
     \item US distributional national accounts data \citep{piketty2018dina}
     \item aggregate to state-year-income-group panel for income, mortgage debt, non-mortgage debt for 1980--2007
     \end{itemize}     
   \end{block}
\pause
   \begin{block}{Answers}
     \begin{enumerate}
     \item Yes.
     \item Yes. \pause The \emph{comparisons channel} generates the findings. The \emph{Saving Glut of the Rich} \citep{mian2020savingglutrich,kumhof2015ineq} does not.
     \end{enumerate}     
   \end{block}

 \end{frame}
 
  \begin{frame}{Findings}
     \begin{itemize}
     \item There is a positive relationship between top incomes and mortgage debt of the non-rich across space (US states) and time (1980--2007)
     \item The relationship is strong and persistent
     \item There is no such relationship for non-mortgage debt
     \end{itemize}

 \end{frame}

 \begin{frame}{Findings 1: Long differences 1980--2007}
   \centering
   \includegraphics[scale=0.17]{figures/long-diff}
 \end{frame}

 \begin{frame}{Findings 2: State-Year Variation - Partial out fixed effects}
   \centering
   \includegraphics[scale=0.17]{figures/state-year}
 \end{frame}

 \begin{frame}{Findings 3: Impulse Responses from Local Projections}
   \centering
   \includegraphics[scale=0.17]{figures/irf}
 \end{frame}

 \begin{frame}{Conclusion}

   \begin{enumerate}
   \item there is a strong and robust positive relationship between top incomes and mortage debt
   \item \emph{Saving Glut of the Rich} is not enough to explain this
     \begin{itemize}
     \item variation across states
     \item difference between mortgage and non-mortgage debt
     \end{itemize}
     \pause
   \item Inequality is an important potential driver of mortgage debt
   \item Social comparisons can generate our findings
   \end{enumerate}
 \end{frame}
 
 \section{Chapter 3 \\ \large{Understanding Housing Wealth Effects: Debt, Home-Ownership and the Lifecycle \\ } \footnotesize{with Frederick Zadow}}


 \subsection{Introduction}
 This dissertation consists of three chapters. Each chapter is self-contained.

\paragraph{Chapter 1} is joint work with Moritz Drechsel-Grau.\footnote{University of Mannheim.}
%\input{../../falling-behind/paper/abstract}

\paragraph{Chapter 2} is joint work with Moritz Drechsel-Grau as well.
%\input{../../inequality-mortgages/paper/abstract}

\paragraph{Chapter 3} is joint work with Frederick Zadow.\footnote{University of Mannheim.}
%\input{../../housing-wealth-effects/paper-old/abstract}



%%% Local Variables:
%%% TeX-engine: luatex
%%% TeX-master: "main"
%%% End:

 \subsection{A Tractable Model of Consumption, Housing and Mortgage Debt}
 \input{../../housing-wealth-effects/slides/model}
 \subsection{Empirics}
 \input{../../housing-wealth-effects/slides/empirics}
 \subsection{Summary}

 \begin{frame}{Summary}
   \begin{itemize}
   \item We provide a tractable model with housing wealth effects
   \item In our model
     \begin{itemize}
     \item housing preferences, age and homeownership drive housing wealth effects
     \item the individual effect of credit is ambiguous
     \end{itemize}
   \end{itemize}
 \end{frame}

 \begin{frame}[allowframebreaks]{Literature}
   \bibliographystyle{ecta} 
   \bibliography{../../header-files/bib/shame}
   % \addcontentsline{toc}{section}{References}
 \end{frame}

\end{document}

%%% Local Variables:
%%% TeX-engine: luatex
%%% TeX-master: t
%%% End:
